\chapter{Business Understanding}
In \cite{CRISPDM2000}, different tasks, and outputs for developting a business understanding are mentioned. The task and respective output will be dicussed in the following sections.

\section{Determine Business Objectives}
Businesses without existing an \ac{ERP} solution in place can easily be overwhelmed by the number of invoices reaching them daily. Even more, the controlling department can easily lose the overview of spending. To quickly gain a perspective on the most important spending topics, spending should be sorted in categories of similar nature.

The primary goal is the development of a solution for automatic aggregation of documents, based on topics adressed in those documents. The focus is on shorter text segments, such as product descriptions. The business objective is an information gain, on how spending is distributed among cross-cutting topics in a company.

The created solution can be evaluated with the business success criterion: "Does the solution identify and give useful insights in the money pits?". This judgement of having achieved the goal is a subjective matter. Evaluating the success should therefore be distributed among several stakeholders, including but not limited to the author, the supervisor and the supplier of the data.


\section{Assess Situation}

\subsection{Inventory of Resources}
An inventory of resources (\ref{tabelle:inventory}) was created for assessing the situation. Most notably is the availability of experts through excellent intercorporational cooperation. Also, a large collection of datasets is available. Hardware platforms include personal machines as well as hosted environments with GPU capabilities. Available software are data science tools included in the Anaconda Navigator, such as Jupyter Notebook. All open-source libraries are of course also included.

\subsection{Requirements, assumptions, and constraints}
The project is to be completed the latest on June 7$^{th}$ 2022. 

Several assumptions underly the process of data mining. First, it is assumed that the descriptions of the invoices is speaking enough to identify the product referenced.
Second, the analysis assumes that invoices can be logically grouped into clusters, in other words, several invoices refering to similar topics.

From a legal perspective, the project is constrained in the publication of data. While the use and processing of the supplied data is permitted within the context of the thesis, publication and further use is prohibited. The dataset is to be kept only on the local machine and SAP owned hyperscaler instances.

\subsection{Risks and contingencies}
One specific risk that may arise in this project, is a dataset too large to be processed. Only a limited size of data is able to be loaded into memory. If the dataset turns out to be too big, four solutions are proposed.

- compress
- data types
- chunking
- AI Core

- data too dirty

- calculations highly inefficient

- clustering does not give speaking results

\subsection{Terminology}

\subsection{Costs and benefits}

\section{Determine data mining goals}
The following data mining goals were identified during the phase of business understanding:

\begin{enumerate}
\item Identifying and applying appropriate methods for feature extracting tailored to this type of dataset.
\item Identifying and applying appropriate methods for clustering documents in this type of dataset.
\item Identifying and applying appropriate methods for topic modelling with this type of dataset.
\item Aggregating expenses by their clusters and visualizing the output.
\end{enumerate}

The successful outcome is defined by reaching all named criteria. The achievment will be evaluated by the author and the supervisor, also people referenced in the inventory of resources will be considered to evaluate the outcome.


\section{Produce Project Plan}

\subsection{Project Plan}

\subsection{Initial assessment of tools and techniques}










