\chapter{Introduction}
\section{Motivation}

More than 4 out of 5 financial departments are "overwhelmed by the high numbers of invoices they are expected to process"  \cite{manualInvoiceProcessing}. Already swamped departments of course struggle with providing information on savings potential. A solution for processing large amounts of invoice data with minimal human interference is desirable. Processing this information should transform unstructured data into a useful format, further the data should be enriched in a way giving insight and provide value for financial advisors. With analysis results of this automated processing, financial advisors can enjoy two improvements. Firstly, they have a factual base for recommendations. Secondly, their financial department is able to focus on more value-adding tasks. If an automated solution completes the time-consuming task of an in-depth analysis of spending, more time is available for interpreting the results.

In general, internal financial reporting aims to provide information about the health of a company, and supply means for improvement. Invoices are the main source of information for controlling \cite{investopediaInvoices}, as they record the complete history of cash flow \cite{invoicesPurpose}. An invoice is a document recording the main information on a sales transaction. Usually, an invoice contains the unit cost, a timestamp, and payment terms. Other information, such as shipping terms, shipping address, or discounts may also be included. 

\section{Current Situation}
An essential part of economic counseling is the assessment of spending for different company segments. Spending of a firm usually is written down in invoice documents, which have to be grouped  to analyze cost types.
The global market is estimated to comprise 550 billion invoices annually, but 90\% are exchanged paper-based \cite{kochEInvoicingJourney}. With modern technology, these paper-based or digital documents can be transformed into a structured or semi-structured format. According to expert estimates, unstructured data makes up for more than 80\% of enterprise data \cite{structuredAndUnstructuredData}. Companies can not utilize unstructured data to its full potential, as this data is not able to be leveraged with traditional data analysis tools.

\section{Research Questions}
\label{section:research-q}
How can information from invoice-like documents be extracted?
How can extracted information provide insights?

\section{Outline}
The introductory chapter briefly explains the motivation behind the thesis. Also, the current situation is assessed and research questions are presented. Lastly, it presents the structure of the thesis.

The following chapter about objectives and criteria gives a detailed task description, and established criteria. In the section \ref{research-model} the process model is explained, evaluated and adjusted.

Chapter 3 gives a glossary of terms. Chapter 3 sheds light on the corporate environment with respect to the aspects of history, organization an technological aspects. 

The following chapters follow the structure of the research model presented in \ref{research-model}.

Chapter 4 explains the process of selecting a dataset. It presents fundamental types of projects and data sources. The chapter explains the sourcing of the used dataset.

Chapter 5 explains the business understanding.

Chapter 6 explains the data understanding.

In chapter 7, the preparation of the data is explained. The process of data cleaning is presented, as well as different means for feature extraction.

Chapter 8 compares algorithms and alternatives for measuring distances.

Chapter 9 presents the evaluation of the result. The chapter evaluated the output from the previous steps.

In chapter 10, a conclusion is drawn, and a further outlook is given.