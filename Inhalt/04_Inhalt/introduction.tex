\chapter{Introduction}
\section{Motivation}

An invoice is a document recording the main information regarding a sales transaction. Usually, an invoice contains the unit cost, a timestamp and the payment terms. Other information, such as shipping terms, shipping adress or payment conditions may also be included.

Apart from their use for tax records, tracking the inventory and legal protection, invoices are essential for a company's external and internal financial reporting (controlling). They are the main source of information for controlling \cite{investopediaInvoices}, as they record the complete financial history of cash flow \cite{invoicesPurpose}. In general, interal financial reporting aims to provide information about the health of a company and supply means for improvement for a company's financials. In turn, this requires an in-depth analysis of spending.

According to \cite{manualInvoiceProcessing}, more than 4 out of 5 financial departments are "overwhelmed by the high numbers of invoices they are expected to process". Already overwhelmed departments of course struggle with providing information on savings potential. A solution for processing large amounts of invoice data with minimal human interference is desirable. With analysis results, financial advisors have a factual base for recommendations.

\section{Current situation}
An essential part of economic counselling is the assessment of allocated spending for different segments of a company. Spending of a firm usually is written down in invoice documents, which have to be grouped for segments to analyze their costs.
While the global market is estimated to comprise 550 billion invoices annually, 90\% are exchanged paper-based \cite{kochEInvoicingJourney}. With modern technology, these paper-based or digital documents can be transformed into a structured or semi-structured format. According to expert estimates, unstructured data makes up for more than 80\% of enterprise data \cite{structuredAndUnstructuredData}. This data is not leverageable with traditional data analysis tools, but its value must be harvested for companies to utilize their full potential. A large share of unstructured data found in companies is textual data.

\section{Research Questions}
Which methods exist for clustering large-scale multilingual corpora? Which combinations of feature-selection methods and clustering methods return the most valuable information?

\section{Outline}
The introductory chapter briefly explains the motivation behind the thesis. Also, the current situation is assessed and research questions are presented. Lastly, it presents the structure of the thesis.

The following chapter about objectives and criteria gives a detailes task description, and established criteria. The section \ref{research-model} the process model is explained, evaluated and adjusted.

Chapter 3 gives a glossary of terms. Chapter 3 sheds light on the corporate environment with respect to the aspects of history, organization an technological aspects. 

The following chapters follow the structure of the research model presented in \ref{research-model}.

Chapter 4 explains the process of selecting a dataset. It presents fundamental types of projects and data sources. The chapter explains the sourcing of the used dataset.