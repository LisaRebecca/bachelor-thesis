\chapter{Objectives and Criteria}
\section{Detailed Task Description}
\label{section:detailed-task}
The goal of this thesis is adding value to real business documents by transforming unstructured data into a structured format. The structured information can then be used as a base for further analyses. 

The task is to perform a full data analysis on the supplied dataset. The dataset is to be prepared for processing with established methods. An evaluation for different means of feature extraction, machine learning, model evaluation and visualization should be performed. With the evaluation a complete flow for the data processing should be presented. The result is an added value to the dataset by creating structured data and providing insights into this data.

\section{Criteria}
\label{section:criteria}
The task includes different criteria from a corporate perspective which need to be considered. The analysis should be performed utilizing only available resources, which are the student’s company laptop and already available instances for SAP internal services. The dataset which will later be presented is only available to SAP employees, and only after an access request for a specific use case is approved. Therefore, the tasks need to be completed with special attention to data protection. 

Additional quality criteria were established before the start of the work. Firstly, the solution should be designed according to industry standards. This also includes the choice for a fitting research model. Secondly, the source code is to be documented in such a way, that an expert third party can understand the workings of it in an appropriate time. Thirdly, the design of the source code should account for existing hardware limitations and should be optimized computationally.

\label{research-model}
\section{Research Model}

To solve the task described in Chapter 1.2, this paper employs the \ac{CRISP-DM} \cite{CRISPDM2000}. This model puts forward a structure for conducting data mining projects. \ac{CRISP-DM} was developed in 1996 by three companies, which are now the partners of the \ac{CRISP-DM} consortium: NCR, DaimlerChrysler AG and SPSS Inc. 

A poll \cite{CRISPDMPopular2020} conducted among visitors of a data science project management blog found that almost half of all respondents employ the \ac{CRISP-DM} process model. Followed by Scrum and Kanban with a 18\% and 12\% of the user share, \ac{CRISP-DM} is by far the most popular. Other methods such as \ac{KDD} and \ac{SEMMA} are also noteworthy alternatives, but are less popular that CRISP-DM, Scrum and Kanban.

\begin{figure}[ht]
	\centering
	\includegraphics[height=10cm]{Bilder/Research_Model.png}
	\caption[Adjusted CRISP-DM Model]{Adjusted CRISP-DM Model, based on \cite[p.~13]{CRISPDM2000}}
	\label{fig:CRISM-DM}
\end{figure}

Being the most popular model, \ac{CRISP-DM} is not necessarily always the best fit for all data science projects. In the case of this particular research effort, \ac{CRISP-DM} proved the best suit for several reasons.
Firstly, the process model follows the natural intuition of project design for data science tasks. Evaluation has to occur before the deployment, the modeling needs to occur before the evaluation, the preparation needs to occur before the modeling, and an adequate understanding of business and data aspects has to be developed in the beginning of the process. All those elementary dependencies are reflected in the model.
Secondly, the \ac{CRISP-DM} model addresses the iterative nature of data mining. Fundamentally, the model is of circular nature, reflecting the fact that data science projects require continuous improvement. After the deployment of one solution, monitoring can give insights which allow for deeper business understanding, triggering the start of a new circuit of the model. Another model which puts forward an iterative approach is \ac{KDD} \cite[p.~41]{KDD}.
Thirdly, the model allows to adapt the order of its phases. The free choice of path is more favorable compared to \ac{KDD}, which allows for loops, but has a fixed order \cite[p.~42]{KDD}.
Fourth, \ac{CRISP-DM} has no special requirements regarding team size or roles. Instead, a \ac{CRISP-DM} project can be completed by only one person. This stands in contrast to SCRUM, which needs different roles represented by people in the team to work effectively \cite{SCRUMSolo}. 
Fifth, the \ac{CRISP-DM} model was publicized in the context of a 70-page guide with generic task descriptions and outputs for each phase. The detailed guide is especially valuable for teams with little experience.

Classically, the reference model consists of six phases. For this thesis the model was adapted to reflect all tasks encompassed. The phase "Dataset Selection" was added, resulting in a total of seven phases. The newly added phase includes the selection of a suitable dataset, the data retrieval, and the data provisioning. This results in a process model adapted to this specific project, and lays the groundwork for a successful undertaking.

The Chapters 4 to 10 each document one of the seven phases of the process. The Chapters 7 and 8 additionally subdivide into an assessment of alternatives, a theoretical and a practical implementation.

 