\chapter{Fundamentals}
\section{Glossary of Terms}

\section{Corporate Environment}
\section{Machine Learning}
Already Alan Turing understood that for laymen a learning machine can be perceived as a paradox.  How can a machine learn, if a human has to define its behavior beforehand? There are three major subfields in the discipline of artificial intelligence that fundamentally explain how a computer can learn how to behave despite predefined behavior.

\subsection{Supervised Learning}
A supervised learning algorithm learns its decision with the help of a data set (input) that also contains the correct decision (output) as information. It is trained with only a part of the entire data set, so that the model can be tested in a later step with the help of unknown data. This way, a statement can be made about the accuracy of the model.

\subsection{Unupervised Learning}
Unsupervised learning is complementary to supervised learning. All algorithms that fall into the category of unsupervised learning are trained with data that does not contain the correct output (label) as information. Here, the categorization is not constrained by the given data, but decided on by the algorithm.

\subsection{Reinforcement Learning}
The third way in which an algorithm can make better decisions as it gains experience is called reinforcement learning. Reinforcement learning is about letting algorithms solve very complex tasks. The special feature is that there is no defined solution path, but the algorithm is rewarded for goal-oriented behavior and punished for wrong decisions. The definition of goal-oriented behavior has to be put into place by the engineers setting up the training of the model. Real-world tasks are extremely complex, so not all possible solution paths can be calculated and compared to find the optimal path. Parking a car is a routine task for a human after a few hours of driving, but a computer sees only an infinite set of possibilities for turning angles. This problem can be solved by reinforcement learning. The algorithm is rewarded for each parking attempt where the car ends up seeing in the parking space. For the remaining attempts, the algorithm is penalized. Over many thousands of attempts, the reinforcement learning model is trained in this way.

The three major ways of learning even with previously defined behavior can now be implemented by specific models. For example, there are several ways to create and train a model using Unsupervised Learning.

\subsection{Clustering Algorithms}
multinomial, one bad example for a clustering would be the closest 5 docs to each one (this is multilabel)

\subsection{Natural Language Processing}
\ac{NLP} is often attributed to the computer science, but after closer examination, \ac{NLP} is a discipline comprised of linguistics, computer science, artificial intelligence an mathematics \cite{gobinda_g_chowdhury_natural_2003}.

\section{SAP AI Core}
\subsection{Docker}
